\documentclass[12pt, letterpaper]{article}

\usepackage{amsmath, amsthm, graphicx, float, verbatim, amssymb}
\newcommand\tab[1][1cm]{\hspace*{#1}}

\usepackage[]{algorithm2e}

\title{Natural Language Processing Homework 5}

\author{Katie Chang}

\begin{document}

\maketitle

README

\section{Simplify (Q2, no need to turn in)}
Mostly just notes for myself

check out d) ($\lambda$a a)($\lambda$b f(b))

f), simplifying ($\lambda$x green(x))(y) = green(y).
Since the result holds for any y, what can you conclude about the relation between $\lambda x green(x)$ and $green$?

Same? $\lambda x green(x)$ applied to anything $y$ means that that something $y$ is green. Similarly, $green$ as a function can be applied to anything $z$ that is green. In any case, they refer to the same set of things??

wait i don't get o)

\section{(Q3) Simplify}

\subsection{John and Mary}
Given $f(John = loves(Mary, John)$

\begin{itemize}  
\item ($\lambda x loves(Mary,x)$)(John)
\item $loves(Mary, John)$ or alternatively, depending on semantics, "Mary loves John" or "John loves Mary".
\end{itemize}

\subsection{John loves Mary}
%In our semantics, loves(Mary,John) will be the interpretation of “John loves Mary,” not vice-versa. This is just more convenient because then the VP in that sentence has a nice, compact semantics. Namely, what?

With the parse tree given for this sentence (as it's in in the semantics slideshow page 52), $\lambda x$ represents "John" and is the NP and $\lambda y$ represents Mary in the VP "loves Mary". If it is flipped, then the parse tree would be different. Furthermore, we'd have to deal with a change in the order we parse. If the meaning is flipped, we'd have to handle $\lambda y$ first, however that is impossible because $\lambda x$ is outside and thus has to be handled first.

\subsection{}
%f(John) = (∀x woman(x) ⇒ loves(x,John))

\begin{itemize}  
\item %what is f?
($\lambda j \forall x woman(x) \Rightarrow loves(x,j)$) 
\item %Translate f and f(John) into English.
Assuming that we will continue with the given semantic that loves(Mary, John) means that John loves Mary.

f : for all x, if x is a woman, then j loves x.

f(John) : for all x, if x is a woman, then John loves x.

\end{itemize}

\subsection{***part d}
%Suppose f (λx loves(Mary, x)) = (λx Obviously(loves(Mary, x))). What is f and how would you use it in constructing the semantics of “Sue obviously loves Mary” Hint: Review the pop/push slide near the end of the semantics lecture.

f = $\lambda y$ Obviously(y)

In order to construct "Sue obviously loves Mary", let y = ($\lambda x loves(Mary, x)$). Then we get 

f($\lambda x loves(Mary, x)$) = 


\subsection{part e}
f = $\lambda m ( \lambda j ( \lambda e$ act(e, loving), lovee(e, m), lover(e, j)))

\subsection{part f}
%Suppose g(f(Mary))(John) = (λe act(e,loving),lovee(e,Mary),lover(e,John),manner(e,passionate)). What is g? Hint: Write out f(Mary), which is the meaning of “loves Mary.” g(f(Mary)) will be the meaning of “passionately loves Mary.” Again, think about the pop/push trick.

\subsection{part g}




\section{Q4}

\section{Q5}


\section{Q6 : english-fullquant.gra}
\subsection{attr}
%The new grammar gives pretty complicated semantic attributes to two and to singular and plural the. Justify the attributes it uses (i.e., explain what those lambda-terms mean). The ! symbol means “not.”

%1 Det[=1 num=pl sem="%dom %pred E%first E%second [first!=second ^ dom(first)^dom(second)] ^ pred(first) ^ pred(second)"] two

For $two$, we are ensuring that the two things that we are quantifying are not the same thing, with the $first$ and $second$ quantifiers on the $dom$ and $pred$ variables. Otherwise, we can end up counting a given something twice, which in reality then doesn't mean that we have two, but rather that we just counted one thing twice. 

%1 Det[=1 num=sing sem="%dom %pred E%t [dom(t) ^ !E%u u!=t ^ dom(u)] ^ pred(t)"] the

The singular $the$ idek
"the book"

%1 Det[=1 num=pl sem="%dom %pred E%T [exhaustive(T,dom)] ^ pred(T)"] the
exhaustive 
"the books"

\subsection{???}

\begin{center}
\textit{Used overleaf.com to generate LaTeX document.}
\end{center}
\end{document}
